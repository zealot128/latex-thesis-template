% FullRef:   siehe \ref{sec:bla}   -> siehe 7 - "Kapitelname"

%%%%%%%%%%%%%%%%%
% Makros fuer den Einsatz von Dingens am Rand
%%%%%%%%%%%%%%%%%
\newcommand{\randbem}[1]{
  \marginpar[\fontsize{9}{10}\selectfont \raggedleft{#1}]
  {\fontsize{9}{10}\selectfont \raggedright{#1}}
}

\newcommand{\bordergraphic}[1]{
  \marginline{\includegraphics[width=0.8\marginparwidth]{#1}}}

\newcommand{\borderquote}[2]{
\setlength{\epigraphwidth}{\marginparwidth}
\marginpar{\fontsize{9}{10} \epigraph{#1}{#2}}
\setlength{\epigraphwidth}{0.8\textwidth}
}


% Einheitliche Bildquelle als zusaetzliches Label
\newcommand{\imgsource}[1]{\captionsetup{font={footnotesize,bf,it}} \caption*{#1}}

% %%% Tiefe des Inhaltsverzeichnis  beschraenken
% \setcounter{tocdepth}{2}	%kleineres TOC

% Langer Pfeil fuer den Fließtext
\newcommand{\arr}{$ \Longrightarrow$}


% Standard-Listings fuer Ruby laden
\lstloadlanguages{Ruby}
\lstset{%
  basicstyle=\ttfamily\footnotesize\color{black},
  commentstyle = \ttfamily\color{gray},
  keywordstyle=\ttfamily\color{blue},
  stringstyle=\color{red},
  showspaces=false,               % show spaces adding particular underscores
  showstringspaces=false,
  frame=single,
  breaklines=true
}
% Umlaute in Listings ermoeglichen
\lstset{literate=%
{Ö}{{\"O}}1
{Ä}{{\"A}}1
{Ü}{{\"U}}1
{ß}{{\ss}}2
{ü}{{\"u}}1
{ä}{{\"a}}1
{ö}{{\"o}}1
}

\newcommand{\glossar}[1]{$^\uparrow$\gls{#1}}
\newcommand{\glossarpl}[1]{$^\uparrow$\glspl{#1}}
