% \fullref{sec:bla}  -> "8.2 - 'Name der Subsection"
% zum Verweis mit Wiederholung des Verweistitels
\newcommand{\fullref}[1]{\ref{#1} -- \glqq\,\textit{\nameref{#1}}\grqq}


% graphik on the right or left border (dep. on twosided)
\newcommand{\bordergraphic}[1]{\marginline{\includegraphics[width=0.8\marginparwidth]{#1}}}

% text on the border
\newcommand{\randbem}[1]{\marginpar[\fontsize{9}{10}\selectfont \raggedleft{#1}]
  {\fontsize{9}{10}\selectfont \raggedright{#1}}
}
% quote on the border
\newcommand{\borderquote}[2]{\setlength{\epigraphwidth}{\marginparwidth}
\marginpar{\fontsize{9}{10} \epigraph{#1}{#2}}
\setlength{\epigraphwidth}{0.8\textwidth}
}


% Einheitliche Bildquelle als zusaetzliches Label
\newcommand{\imgsource}[1]{\captionsetup{font={footnotesize,bf,it}} \caption*{#1}}

% convenient long arrow
\newcommand{\arr}{$ \Longrightarrow$}


% load ruby as a standard listing
\lstloadlanguages{Ruby}
\lstset{%
  basicstyle=\ttfamily\footnotesize\color{black},
  commentstyle = \ttfamily\color{gray},
  keywordstyle=\ttfamily\color{blue},
  stringstyle=\color{red},
  showspaces=false,               % show spaces adding particular underscores
  showstringspaces=false,
  frame=single,
  breaklines=true
}
% Allow umlauts in lstlistings
\lstset{literate=%
{Ö}{{\"O}}1
{Ä}{{\"A}}1
{Ü}{{\"U}}1
{ß}{{\ss}}2
{ü}{{\"u}}1
{ä}{{\"a}}1
{ö}{{\"o}}1
}

% fancy glossar reference with arrow in front of
\newcommand{\glossar}[1]{$^\uparrow$\gls{#1}}
\newcommand{\glossarpl}[1]{$^\uparrow$\glspl{#1}}

  % make table beautiful:
  %   zebra colors, more difference, header bg color
  % cannot be earlier, because frontpage uses tables for positioning
\newcommand{\fancytables}{\rowcolors{1}{lightgray!50!white}{white}
\colorlet{tableheadcolor}{gray!40}
\renewcommand\arraystretch{1.25}% (MyValue=1.0 is for standard spacing)
}
